\section*{Introduction}
\addcontentsline{toc}{section}{Introduction}
Two-dimensional triangular lattice antiferromagnetic systems are a perfect 
representation of geometrically frustrated magnetism \cite{AFM_intro}. The 
magnetic phase diagram of a two-dimensional triangular lattice 
antiferromagnetic system has been studied earlier theoretically, revealing the 
coplanar phase 120$^{\circ}$ phase in the ground state that evolves into Y, up-up-down (UUD), 
and V phase upon increasing magnetic field and temperature
\cite{AFM_intro2}. \\
There are not many two-dimensional antiferromagnetic systems that exhibit 
magnetic phase diagram proposed. In practice, not as many two-dimensional 
triangular lattice Heisenberg antiferromagnetic systems follow the proposed 
model and  different perturbations can destroy or modify the ideal 
ground state, such as anisotropy \cite{anisotropy}, next-nearest-neighbor 
\cite{NNN} or inter-layer interactions \cite{NN_layer}. Compounds 
Na$_2$BaX(PO$_4$)$_2$ (X = Co, Ni, Mn) were reported to show frustrated 
magnetism on the triangular lattice with spins 1/2, 1, and 5/2 respectively. All 
three compounds host the UUD phase \cite{Na2BaCo(PO4)2_phase_diagram, 
NBNPO_phase_diagram, NBMPO_phase_diagram} and their phase diagrams show 
promising conditions for hosting the quantum spin liquid phase. To prove or 
disprove the quantum spin liquid phase one must measure the magnetic 
excitations of the ground state using inelastic neutron scattering, which proves 
to be experimentally very complicated.\\
Single crystals of compounds Na$_2$BaX(PO$_4$)$_2$ (X = Co, Ni, Mn) are 
generally very small with an average mass of 10 mg, which is not enough for 
inelastic neutron experiment. Traditionally, experimental physicists are 
coaligning big amounts of single crystals in order to have
higher inelastic signal \cite{krystaly_coaligment}. This process is very 
time-consuming, demanding, and could suffer from low precision\footnote{Private communications with Huiqian Luo, one of the authors of \cite{krystaly_coaligment}}. For this reason, we developed a new experimental technique for crystal coalignment called Automatic Laue Sample Aligner (ALSA). ALSA combines a robotic arm, camera vision, and Laue diffractometer to coalign crystals automatically. The whole coalignment process is one of the key parts of this thesis.\\
% zpet k fyzice a motivaci vzorku
ALSA provides not only acceleration of the production process of samples for inelastic neutron experiments but also improvement in the quality and precision of the production process. Using ALSA, it would be possible to distinguish complicated spin wave dispersions that were previously hidden in the elastic signal. In this thesis, we present inelastic neutron measurements on multiple samples prepared by ALSA, demonstrating its capabilities. 







