\section{Theory}\label{Theory}
\subsection{Magnetism}\label{Magnetism}
The magnetic moment in magnetism is the fundamental object, which can be defined as \cite{blundel}
\begin{equation}\label{eq:magnetization_def}
    \overrightarrow{d\mu} = I \overrightarrow{dS}
\end{equation}
where $I$ is current going through the oriented loop of area $\overrightarrow{dS}$ and magnetic moment $\mu$ has units of Am$^{2}$. 
In solid-state physics, it is much more convenient to use the Bohr magneton since it describes the magnetic moment at the atomic level. 
It is described as  
\begin{equation}
    \mu_B = \frac{e\hbar}{2m_e}
\end{equation}
A magnetic solid consists of a large number of these current loops with magnetic moments. 
The magnetization $\overrightarrow{M}$ is defined as the magnetic moment per unit volume and defines the vector field inside the solid. 
In the special case  magnetization $\overrightarrow{M}$ is linearly related to the magnetic field $\overrightarrow{H}$, 
the solid is called a linear material, and we can write 
\begin{equation}\label{eq:linear_magnetization}
    \overrightarrow{M} = \chi \overrightarrow{H}
\end{equation}
where $\chi$ is the dimensionless quantity called the magnetic susceptibility. \\
The electronic angular momentum  defined by equation \ref{eq:magnetization_def} is associated with the orbital motion of an electron around the nucleus. 
In a real atom, it depends on the electronic state occupied by the electron. 
In addition, the electron possesses an intrinsic magnetic moment which is associated with intrinsic angular momentum. 
This intrinsic angular momentum of an electron is called spin. 
The magnetic moment on an atom is associated with its total angular momentum $\overrightarrow{J}$ which is determined by the third Hund's rule \cite{Magnetism_total_momentum}. 
We can calculate total angular momentum from orbital angular momentum $\overrightarrow{L}$ and the spin angular momentum $\overrightarrow{S}$.